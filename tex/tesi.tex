% Document Class and packages
\documentclass[a4paper]{memoir}
\usepackage{subfiles}
\usepackage[utf8]{inputenc}
\usepackage[italian]{babel}
\usepackage{graphicx}
\usepackage[usenames,dvipsnames,svgnames,table]{xcolor}
\usepackage{geometry}
\usepackage{cite}

% about:author
\author{Paolo Scaramuzza}
\title{Studio dell' efficienza di oscillatori LC integrati per impulsatori 
	Ultra Wideband}
\date{} % HACK: avoids date in the title

\begin{document}
\subfile{tex/front} % make a nice front page (thanks HammondSmoking)
\cleardoublepage{}
\newpage

%%%%%%%%%%%%%%%%%%%%%%%%%%%%%
% Abstract
%%%%%%%%%%%%%%%%%%%%%%%%%%%%%
\pagenumbering{gobble} % no need for numbering
\begin{vplace}[0.7]
\begin{abstract} % TODO: ~70 more words!
Gli impulsatori Ultra-Wideband (\emph{Ultra-Wideband Impulse-Radio}, UWB IR)
originariamente nati e impiegati in ambito militare, cominciano ad emergere
anche in applicazioni civili.
Grazie alla regolamentazione, in particolare da parte di USA Europa e Giappone 
dello spettro di frequenze tra 3.1 e 10.6 GHz, trasmettitori e ricevitori UWB
IR vengono impiegati soprattutto per trasmissioni a corto raggio e
localizzazione di precisione.

In questa tesi si analizza e dimensiona un oscillatore LC in topologia
\emph{cross-coupled} operante alla frequenza di 8Ghz e realizzato in tecnologia
CMOS da $13 \mu m$.

Utilizzando il simulatore cicuitale Cadence Design Suite (E' il nome corretto?
che versione??) ed alcuni modelli teorici, si studia l' effetto di variazioni
nel dimensionamento dei componenti sul comportamento del circuito, ponendo
particolare attenzione all' efficienza energetica della trasmissione.
Si propone infine la soluzione che massimizza la potenza all' antenna
rispetto a quella dissipata nel cicruito dai componenti parassiti.
\end{abstract}
\end{vplace}

%%%%%%%%%%%%%%%%%%%%%%%%%%%%%
% Index
%%%%%%%%%%%%%%%%%%%%%%%%%%%%%
\cleardoublepage{}
\newpage
\tableofcontents

\pagenumbering{arabic} % restore normal page numbering
%%%%%%%%%%%%%%%%%%%%%%%%%%%%%
% Introduzione
%%%%%%%%%%%%%%%%%%%%%%%%%%%%%
\cleardoublepage{}
\chapter{Introduzione}

%%%%%%%%%%%%%%%%%%%%%%%%%%%%%
% Lo stato dell' arte
%%%%%%%%%%%%%%%%%%%%%%%%%%%%%
\cleardoublepage{}
\chapter{Lo stato dell' arte}

%%%%%%%%%%%%%%%%%%%%%%%%%%%%%
% La simulazione
%%%%%%%%%%%%%%%%%%%%%%%%%%%%%
\cleardoublepage{}
\chapter{La simulazione}
\cite{Razavi11}
%%%%%%%%%%%%%%%%%%%%%%%%%%%%%
% Risultati numerici
%%%%%%%%%%%%%%%%%%%%%%%%%%%%%
\cleardoublepage{}
\chapter{Risultati numerici}
\includegraphics[width=\textwidth]{efficienza.pdf}
%%%%%%%%%%%%%%%%%%%%%%%%%%%%%
% Conclusione e sviluppi futuri
%%%%%%%%%%%%%%%%%%%%%%%%%%%%%
\cleardoublepage{}
\chapter{Conclusione}

\section{Sviluppi futuri}

%%%%%%%%%%%%%%%%%%%%%%%%%%%%%
% Bibliografia
%%%%%%%%%%%%%%%%%%%%%%%%%%%%%
\bibliographystyle{plain}
\bibliography{tex/biblio}
\end{document}
