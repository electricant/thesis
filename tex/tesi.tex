% Document Class and packages
\documentclass[a4paper, 12pt]{memoir}
\usepackage{subfiles}
\usepackage[utf8]{inputenc}
\usepackage[italian]{babel}
\usepackage{microtype}
\usepackage{graphicx}
\usepackage[usenames,dvipsnames,svgnames,table]{xcolor}
\usepackage[a4paper, bindingoffset=1cm]{geometry}
\usepackage{cite}
\usepackage{hyperref}

% Increase space between paragraphs
\setlength{\parskip}{3pt}
% increase space between lines
\linespread{1.2}

% about:author
\author{Paolo Scaramuzza}
\title{Studio dell' efficienza di oscillatori LC integrati per impulsatori 
	Ultra Wideband}
\date{} % HACK: avoids date in the title

\begin{document}
\subfile{tex/front} % make a nice front page (thanks HammondSmoking)

%%%%%%%%%%%%%%%%%%%%%%%%%%%%%
% Abstract
%%%%%%%%%%%%%%%%%%%%%%%%%%%%%
\cleardoublepage{}
\newpage
\pagenumbering{gobble} % no need for numbering
\begin{vplace}[0.7]
	\subfile{tex/abstract}
\end{vplace}
%%%%%%%%%%%%%%%%%%%%%%%%%%%%%
% Index
%%%%%%%%%%%%%%%%%%%%%%%%%%%%%
\cleardoublepage{}
\newpage
\tableofcontents

\pagenumbering{arabic} % restore normal page numbering
%%%%%%%%%%%%%%%%%%%%%%%%%%%%%
% Introduzione
%%%%%%%%%%%%%%%%%%%%%%%%%%%%%
\chapter{Introduzione}
Contrariamente a molte tecnologie in radiofrequenza oggi in uso, la radio a
impulsi Ultra-Wideband (\emph{Ultra-Wideband Impulse-Radio}, UWB IR)
sfrutta un' ampia porzione dello spettro delle radiofrequenze (di ampiezza
maggiore a 500MHz oppure più del 25\% della frequenza di centro-banda
\cite{Neviani12}) trasmettendo i dati sotto forma di impulsi di breve durata.\\
Tra gli schemi di modulazione impiegati si trovano: 
\begin{itemize}
	\item \emph{on-off keyng} (OOK)
	\item \emph{pulse-position modulation} (PPM)
\end{itemize}

Per molti anni la UWB IR è stata impiegata esclusivamente in ambito
militare. Nel 2002, con la regolamentazione dello spettro di frequenze tra 3.1 
e 10.6GHz da parte degli Stati Uniti e in seguito dall'Europa e dal Giappone,
la radio UWB ha iniziato ad affermarsi quale soluzione più conveniente, sia dal
punto di vista economico che prestazionale, per la comunicazione a corto raggio.
\\La massima densità spettrale di potenza in trasmissione consentita dalle
leggi europee è di $ -41.3 dBm/MHz $ e in alcuni segmenti dello spettro tale
limite può arrivare fino a $ -75 dBm/MHz $, come evidenziato in\cite{Hirt07}.\\
Vista la bassa potenza di trasmissione e le elevate frequenze operative,
le radio a impulsi UWB ben si prestano ad essere realizzate interamente in
tecnologia CMOS.\@
Si possono avere così trasmettitore e ricevitore in un unico circuito integrato.
\\Tra le possibili applicazioni di questa tecnologia si trovano: la
localizzazione di precisione e la trasmissione a corto raggio.

Per la localizzazione si cita\cite{Terada05} in cui si mostra una radio 
Ultra-Wideband realizzata in tecnologia CMOS da $ 180nm $ in grado di
effettuare localizzazioni a più di un metro di distanza con un errore massimo
di $ \pm 2.5cm $.
Quando si effettuano mille misurazioni al secondo il consumo di potenza è pari
a $ 0.7\mu W $. Il trasmettitore e il ricevitore sono integrati in un unico
chip avente un' area di $ 0.415mm^2 $.

Per la trasmissione a corto raggio significativi risultati sono stati raggiunti
da\cite{RabaeyEECS},\cite{Neviani12} e\cite{Gambini12} in cui sono dimostrate
varie soluzioni circuitali che permettono di ottenere, in un unico circuito
integrato di piccole dimensioni, sia la funzionalità di trasmettitore che di
ricevitore.

Attualmente la frontiera della ricerca si sta spostando verso una sempre
maggiore integrazione e un miglioramento globale dell'efficienza energetica.
Questi risultati della Ricerca permettono di avere radio dalla durata della 
batteria estremamente lunga o di far funzionare la circuiteria tramite
\emph{energy harvesting}.\\
Un esempio significativo è\cite{Danesh11} in cui si ha un nodo sensore 
alimentato da un pannello solare di $ 2\times2 cm^2 $ che svolge anche la
funzione di antenna. Il sensore è poco più grande del pannello solare, consuma
solamente $ 10\mu W $ di potenza e trasmette i dati a intervalli di un
minuto alla velocità di $ 1 KB/s $ tramite la modulazione \emph{on-off
keying}.
Con l' impiego di un supercondensatore, inoltre, è in grado di lavorare in
condizioni di totale assenza di illuminazione per più di due giorni.\\
In\cite{Solda10} infine è dimostrata la realizzazione di una radio a impulsi 
che garantisce un' efficienza in trasmissione pari al 7\% mentre in 
\cite{Neviani14} si raggiunge l' 11,7\%, valore che rappresenta l' attuale
stato dell'arte.

La presente tesi parte proprio dai risultati ottenuti in~\cite{Neviani12} e
\cite{Neviani14} e mira a massimizzare l'efficienza energetica dell'oscillatore
LC \emph{cross-coupled} in uso nell'impulsatore.

In Figura 1.1 è rappresentato lo schema elettrico di un oscillatore LC
\emph{cross coupled} nella sua forma più semplice.\\
\begin{figure}[h]
\centering
\includegraphics[width=0.34\textwidth]{images/LCosc.pdf}
\caption{Schema elettrico dell'oscillatore LC \emph{cross-coupled}}
\end{figure}
\`E la tipologia di oscillatore più diffusa nei circuiti integrati perché il
rumore di fase è più basso rispetto ad altre soluzioni e richiede un basso
numero di componenti attivi\cite{Amran05}\cite{RazaviRF}.

Il suo funzionamento si basa sulla risonanza tra l'induttanza e la capacità
presenti nel circuito. \\
Gli NMOS del circuito infatti formano una coppia incrociata e ognuno dei due
rami apporta, come si evince dai diagrammi di Bode in Figura 1.2, uno
sfasamento di $ 180^{\circ} $ alla pulsazione $ \omega _r=\frac{1}{\sqrt{LC}} $.

Per il singolo ramo del circuito, detta $ R_p $ la resistenza parassita
complessiva e $ g_m $ la transconduttanza del MOSFET, si ha la seguente
funzione di trasferimento:
\begin{center}
$ W(s)=\frac{V_{out}}{V_{in}}=-g_m Z(s)=-g_m \frac{R_p L s}{R_p C L s^2 + Ls + R_p} $
\end{center}

\begin{figure}[h]
\centering
\includegraphics[width=\textwidth]{images/BodeLC.pdf}
\caption{Diagramma di Bode per $ Z(s) $ con $L=1H$, $C=1F$ e $R_p=1K\Omega$}
\end{figure}

In considerazione della $ W(s) $ sopra descritta è possibile applicare il
criterio di Barkhausen\cite{JaegerMicro} che afferma che il circuito si
comporta da oscillatore se alla pulsazione $ \omega _r $ il modulo della
funzione di trasferimento in catena aperta è pari ad uno e lo sfasamento è di
$ 360^{\circ} $.\\
Per avere oscillazione è dunque necessario porre in cascata, chiudendo l'
anello di retroazione, due stadi del tipo di circuito presentato in Figura 1.3,
assicurandosi che $ {\left( g_m R_p \right)}^2 = 1 $ 
\cite[p.652]{RazaviFundamentals}.\\
Riorganizzando lo schema circuitale si riottiene quello in Figura 1.1.

\begin{figure}[h]
\centering
\includegraphics[width=0.5\textwidth]{images/LCsingle.pdf}
\caption{Schema per il singolo ramo dell'oscillatore}
\end{figure}

\section{Contributi della tesi}
Nell'ambito del lavoro svolto i valori dell'induttanza e della capacità sono
tarati in modo tale da produrre un' onda sinusoidale alla frequenza di 8GHz.\\
L' induttanza è sostituita da un trasformatore al cui secondario è collegata l'
antenna. La circuiteria di controllo aggiunta si occupa di attivare e
disattivare l' oscillatore per produrre gli impulsi.

Un modello teorico semplificato per il circuito è stato ricavato approssimando
i MOSFET come dei generatori ideali di tensione alternata ad 8GHz
connessi al primario del trasformatore. \\
Tale modello ha permesso di valutare in prima approssimazione l' effetto sul
comportamento del circuito delle variazioni nel dimensionamento dei componenti.

Partendo dai valori ottenuti in\cite{Neviani14} si è poi effettuata la
simulazione al calcolatore del circuito a livello di transistor.\\
Si è evidenziato che, a meno di effetti di ordini successivi al primo, i
risultati della simulazione sono in accordo con il modello teorico.\\
Affinando le indicazioni del modello tramite il calcolatore, si sono
raggiunti valori di efficienza massima per il circuito simulato nell'intorno
del 29\%, partendo da un valore iniziale pari al 10\%.\\
Le simulazioni effettuate su trasformatore da solo confermano ulteriormente i
risultati ottenuti.

Si presentano e discutono infine i massimi valori di efficienza raggiunti nelle
simulazioni dalle due topologie ed è illustrato come queste differiscano poco l'
una dall'altra dal punto di vista dell'efficienza. \\
La scelta di un tipo di oscillatore piuttosto che l' altro deve dunque essere
dettata da altri parametri (ad esempio l' occupazione di area ed il consumo
dinamico di potenza) che esulano dal contributo di questa tesi e sono solo
accennati.
\newpage
\section{Struttura dell'elaborato}
Il presente elaborato è organizzato come segue:
\begin{itemize}
\item nel Capitolo 2 sono presentate le due diverse tipologie di oscillatore
	prese in esame. \`E discusso come da essi si giunga ad un modello
	semplificato e si presenta un'analisi qualitativa degli effetti
	sull'efficienza del diverso dimensionamento dei componenti nel
	modello ottenuto.
\item Nel Capitolo 3 sono presentati e discussi i risultati numerici delle
	simulazioni, confrontando le diverse topologie dell'oscillatore.
\item Il Capitolo 4 è dedicato agli sviluppi futuri e alla discussione dei
	parametri di interesse la cui valutazione si presta ad ulteriori studi.
\end{itemize}
%%%%%%%%%%%%%%%%%%%%%%%%%%%%%
% Tipologie e parametri
%%%%%%%%%%%%%%%%%%%%%%%%%%%%%
\chapter{Topologie e parametri di interesse}
Come è stato descritto nel capitolo precedente, sono stati analizzati due
diversi oscillatori LC in topologia \emph{cross coupled} per la realizzazione di
impulsatori ultra-wideband.

\section{Le topologie}
In Figura 2.1 è presentato lo schema dell'oscillatore denominato di Tipo 1.\\
Alla topologia basilare è aggiunto un NMOS (indicato con M3) con
drain e source connessi tra l' oscillatore e la massa. L'accensione e lo
spegnimento avvengono fornendo al gate di M3 un valore logico rispettivamente
alto e basso.\\
\begin{figure}[h]
\centering
\includegraphics[width=0.67\textwidth]{images/tipo1.pdf}
\caption{Schema elettrico dell'oscillatore di Tipo 1. Si noti l' NMOS M3, la
      cui funzione è quella di connettere o disconnettere il circuito
	dall'alimentazione.}
\end{figure}
Quando M3 è spento, ovvero quando la tensione tra gate e drain è
nulla, esso si comporta come un circuito aperto e dunque l'oscillatore non
viene alimentato. Quando invece M3 è in saturazione, ovvero quando la tensione 
al gate supera la tensione di soglia, la sua resistenza è molto bassa e dunque 
il circuito viene chiuso e l'oscillatore inizia a funzionare.

In Figura 2.2 è presentato lo schema dell'oscillatore denominato di Tipo 2.\\
Rispetto alla soluzione precedente una coppia di invertitori in logica
complementare è connessa ai gate degli NMOS dell'oscillatore. L'accensione e
lo spegnimento avvengono quando all'ingresso degli invertitori si presenta un
valore logico basso o alto.
\begin{figure}[h]
\centering
\includegraphics[width=0.75\textwidth]{images/tipo2.pdf}
\caption{Schema elettrico dell'oscillatore di Tipo 2. Gli invertitori CMOS
      connessi ai gate di M1 e M2 attivano o cortocircuitano a massa le
      oscillazioni.}
\end{figure}
\clearpage % leave space for the pictures
\noindent Quando \texttt{STOP} è alto sono attivi gli NMOS degli invertitori
che portano i gate di M1 e M2 a massa, impedendo all'oscillatore di funzionare.
Quando \texttt{STOP} è basso invece sono i PMOS ad essere attivi. I gate di M1
e M2 sono ora connessi a Vdd tramite una resistenza di dimensioni dell'ordine
della decina di Kiloohm e dunque l'oscillatore viene attivato.

\section{Il trasformatore}
Date le topologie di oscillatore presentate nella sezione precedente, il
trasformatore è l'elemento che più incide sull'efficienza della trasmissione.
Esso svolge infatti il ruolo fondamentale di trasformare, senza l'impiego di
ulteriori componenti attivi, la tensione differenziale presente al primario in
quella adatta a pilotare l'antenna\cite{Neviani14}.

Nei circuiti integrati induttanze e trasformatori sono realizzati come
spirali metalliche nello strato di \emph{metal} più superficiale (che è anche
lo strato più spesso).\\
A causa di questa struttura fisica si hanno dei componenti parassiti tra i
quali si annoverano\cite[pp. 431-455]{RazaviRF}:
\begin{description}
\item[capacità] tra le spire e il substrato, tra le spire e gli strati di
	\emph{metal} adiacenti e tra le spire stesse;
\item [resistori] dovuti alla resistenza serie del metallo che compone
	l'induttanza, all'effetto pelle e all'accoppiamento capacitivo con il
	substrato;
\item [induttori] dovuti all'accoppiamento magnetico con il substrato e con le 
	piste adiacenti.
\end{description}
Tutti questi parametri concorrono a degradare il fattore di qualità
dell'induttore.

Per effettuare le simulazioni è stato necessario produrre un modello per il
trasformatore che tenesse adeguatamente conto degli effetti sopracitati.\\
Si è valutato di restringere il campo ai soli elementi resistivi al fine di
ottenere dei dati semplificati utilizzabili per il confronto delle due
tipologie di oscillatore. L' inclusione degli altri elementi parassiti può
rappresentare uno stimolo per ulteriori studi e per affinare l' efficienza una
volta determinata la soluzione circuitale migliore.
\begin{figure}[h]
\centering
\includegraphics[height=0.2\textheight]{images/ic_inductor.JPG}
\caption{Microfotografia di un induttore in un circuito integrato
	\hspace{\textwidth} % force newline
	\footnotesize{Fonte: \url{http://people.seas.harvard.edu/~jones/es154
	/lectures/lecture_0}}}
\end{figure}

Il modello ottenuto per il trasformatore è presentato in Figura 2.4.
Tutti gli effetti indesiderati sono riassunti da una resistenza in serie al
primario divisa a metà tra i due rami. Tale resistenza è calcolata di volta in
volta, a partire dal valore di induttanza al primario, in modo da mantenere il
fattore di qualità Q a 25 per una frequenza di 8GHz.\\
La formula che lega il fattore di qualità al valore di $L_1$ è:
$ Q = 2\pi f_{osc} \frac{L_1}{R_p} $,\\
dove $R_p$ è la resistenza parassita al primario e $ f_{osc}=8GHz $.
\begin{figure}[h]
\centering
\includegraphics[height=0.34\textheight]{images/trasf_model.pdf}
\caption{Modello di trasformatore usato nelle simulazioni}
\end{figure}
\clearpage % pictures please
\section{Il modello teorico}

%%%%%%%%%%%%%%%%%%%%%%%%%%%%%
% Risultati numerici
%%%%%%%%%%%%%%%%%%%%%%%%%%%%%
\cleardoublepage{}
\chapter{Risultati numerici}
\includegraphics[width=\textwidth]{efficienza.pdf}

%%%%%%%%%%%%%%%%%%%%%%%%%%%%%
% Conclusione e sviluppi futuri
%%%%%%%%%%%%%%%%%%%%%%%%%%%%%
\cleardoublepage{}
\chapter{Conclusione}

\section{Sviluppi futuri}

%%%%%%%%%%%%%%%%%%%%%%%%%%%%%
% Bibliografia
%%%%%%%%%%%%%%%%%%%%%%%%%%%%%
\bibliographystyle{plain}
\bibliography{tex/biblio}
\end{document}
