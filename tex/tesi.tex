% Document Class and packages
\documentclass[a4paper, 12pt]{memoir}
\usepackage{subfiles}
\usepackage[utf8]{inputenc}
\usepackage[italian]{babel}
\usepackage{graphicx}
\usepackage[usenames,dvipsnames,svgnames,table]{xcolor}
\usepackage{geometry}
\usepackage{cite}

% about:author
\author{Paolo Scaramuzza}
\title{Studio dell' efficienza di oscillatori LC integrati per impulsatori 
	Ultra Wideband}
\date{} % HACK: avoids date in the title

\begin{document}
\subfile{tex/front} % make a nice front page (thanks HammondSmoking)

%%%%%%%%%%%%%%%%%%%%%%%%%%%%%
% Abstract
%%%%%%%%%%%%%%%%%%%%%%%%%%%%%
\cleardoublepage{}
\newpage
\pagenumbering{gobble} % no need for numbering
\begin{vplace}[0.7]
	\subfile{tex/abstract}
\end{vplace}
%%%%%%%%%%%%%%%%%%%%%%%%%%%%%
% Index
%%%%%%%%%%%%%%%%%%%%%%%%%%%%%
\cleardoublepage{}
\newpage
\tableofcontents

\pagenumbering{arabic} % restore normal page numbering
%%%%%%%%%%%%%%%%%%%%%%%%%%%%%
% Introduzione
%%%%%%%%%%%%%%%%%%%%%%%%%%%%%
\cleardoublepage{}
\chapter{Introduzione}
Contrariamente a molte tecnologie in radiofrequenza oggi in uso, la radio a
impulsi Ultra-Wideband (\emph{Ultra-Wideband Impulse-Radio}, UWB IR)
sfrutta un' ampia porzione dello spettro delle radiofrequenze (di ampiezza
maggiore a 500MHz oppure più del 25\% della frequenza di centro-banda
\cite{Neviani12}) trasmettendo i dati sotto forma di impulsi di breve durata.\\
Tra gli schemi di modulazione impiegati si trovano: 
\begin{itemize}
	\item \emph{on-off keyng} (OOK)
	\item \emph{pulse-position modulation} (PPM)
\end{itemize}

Per molti anni la UWB IR è stata impiegata esclusivamente in ambito
militare. Nel 2002, con la regolamentazione dello spettro di frequenze tra 3.1 
e 10.6GHz da parte degli Stati Uniti e in seguito dall'Europa e dal Giappone,
la radio UWB ha iniziato ad affermarsi quale soluzione più conveniente, sia dal
punto di vista economico che prestazionale, per la comunicazione a corto raggio.
\\La massima densità spettrale di potenza in trasmissione consentita dalle
leggi europee è di $ -41.3 dBm/MHz $ e in alcuni segmenti dello spettro tale
limite può arrivare fino a $ -75 dBm/MHz $, come evidenziato in\cite{Hirt07}.\\
Vista la bassa potenza di trasmissione e le elevate frequenze operative,
le radio a impulsi UWB ben si prestano ad essere realizzate interamente in
tecnologia CMOS.\@
Si possono avere così trasmettitore e ricevitore in un unico circuito integrato.
\\Tra le possibili applicazioni di questa tecnologia si trovano: la
localizzazione di precisione e la trasmissione a corto raggio.

Per la localizzazione si cita\cite{Terada05} in cui si mostra una radio 
Ultra-Wideband realizzata in tecnologia CMOS da $ 180nm $ in grado di
effettuare localizzazioni a più di un metro di distanza con un errore massimo
di $ \pm 2.5cm $.
Quando si effettuano mille misurazioni al secondo il consumo di potenza è pari
a $ 0.7\mu W $. Il trasmettitore e il ricevitore sono integrati in un unico
chip avente un' area di $ 0.415mm^2 $.

Per la trasmissione a corto raggio significativi risultati sono stati raggiunti
da\cite{RabaeyEECS},\cite{Neviani12} e\cite{Gambini12} in cui sono dimostrate
varie soluzioni circuitali che permettono di ottenere, in un unico circuito
integrato di piccole dimensioni, sia la funzionalità di trasmettitore che di
ricevitore.

Attualmente la frontiera della ricerca si sta spostando verso una sempre
maggiore integrazione e un miglioramento globale dell'efficienza energetica.
Questi risultati della Ricerca permettono di avere una durata della batteria che
alimenta il circuito estremamente lunga o di far funzionare le radio tramite
\emph{energy harvesting}.\\
Un esempio significativo è\cite{Danesh11} in cui si ha un nodo sensore 
alimentato da un pannello solare di $ 2\times2 cm^2 $ che svolge anche la
funzione di antenna. Il sensore è poco più grande del pannello solare, consuma
solamente $ 10\mu W $ di potenza e trasmette i dati a intervalli di un
minuto alla velocità di $ 1 KB/s $ tramite la modulazione \emph{on-off
keying}.
Con l' impiego di un supercondensatore, inoltre, è in grado di lavorare in
condizioni di totale assenza di illuminazione per più di due giorni.\\
In\cite{Solda10} infine è dimostrata la realizzazione di una radio a impulsi 
che garantisce un' efficienza in trasmissione pari al 7\% mentre in 
\cite{Neviani14} si raggiunge l' 11,7\%, valore che rappresenta l' attuale
stato dell'arte.

La presente tesi parte proprio dai risultati ottenuti in~\cite{Neviani12} e
\cite{Neviani14} e mira a massimizzare l'efficienza energetica dell'oscillatore
LC \emph{cross-coupled} in uso nell'impulsatore.

%%%%%%%%%%%%%%%%%%%%%%%%%%%%%
% Tipologie e parametri
%%%%%%%%%%%%%%%%%%%%%%%%%%%%%
\cleardoublepage{}
\chapter{Topologie e parametri di interesse}
\cite{Razavi11}

%%%%%%%%%%%%%%%%%%%%%%%%%%%%%
% Risultati numerici
%%%%%%%%%%%%%%%%%%%%%%%%%%%%%
\cleardoublepage{}
\chapter{Risultati numerici}
\includegraphics[width=\textwidth]{efficienza.pdf}

%%%%%%%%%%%%%%%%%%%%%%%%%%%%%
% Conclusione e sviluppi futuri
%%%%%%%%%%%%%%%%%%%%%%%%%%%%%
\cleardoublepage{}
\chapter{Conclusione}

\section{Sviluppi futuri}

%%%%%%%%%%%%%%%%%%%%%%%%%%%%%
% Bibliografia
%%%%%%%%%%%%%%%%%%%%%%%%%%%%%
\bibliographystyle{plain}
\bibliography{tex/biblio}
\end{document}
