%%
% Abstract
%%
\documentclass[tex/tesi.tex]{subfiles}

\begin{document}
\begin{abstract}
Gli impulsatori Ultra-Wideband (\emph{Ultra-Wideband Impulse-Radio}, UWB IR)
originariamente nati e impiegati in ambito militare, cominciano ad emergere
anche in applicazioni civili.
A partire dal 2002, quando lo spettro di frequenze tra 3,1 e 10,6GHz è stato
regolamentato negli USA e poco dopo anche in Europa e Giappone, trasmettitori
e ricevitori UWB IR vengono impiegati in particolare per trasmissioni a corto
raggio e localizzazione di precisione.
Queste applicazioni richiedono una bassa potenza di trasmissione e dunque
circuiti che realizzano impulsatori UWB si prestano bene ad essere integrati
tramite tecnolgie a basso costo come lo standard CMOS.\@

In questa tesi si analizzano e dimensionano due oscillatori LC in topologia
\emph{cross-coupled} operanti alla frequenza di 8Ghz e realizzati in tecnologia
CMOS da $0,13 \mu m$ per la precedentemente citata applicazione.
Tali oscillatori differiscono esclusivamente per la soluzione adottata per l'
accensione e lo spegnimento del circuito tramite in segnale di controllo
esterno.

Utilizzando il simulatore cicuitale Cadence Design Suite (E' il nome corretto?
che versione??) ed alcuni modelli teorici, si studia l' effetto di variazioni
nel dimensionamento dei componenti sul comportamento del circuito, ponendo
particolare attenzione all' efficienza energetica della trasmissione.
Si propone infine la soluzione che massimizza la potenza all' antenna rispetto
a quella dissipata nel cicruito dai componenti parassiti.
\end{abstract}
\end{document}
