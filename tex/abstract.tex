%%
% Abstract
%%
\documentclass[tex/tesi.tex]{subfiles}

\begin{document}
\begin{abstract}
La radio a impulsi Ultra-Wideband (\emph{Ultra-Wideband Impulse-Radio}, UWB IR)
originariamente nata e impiegata in ambito militare, comincia ad emergere
anche in applicazioni civili.
A partire dal 2002, quando lo spettro di frequenze tra 3,1 e 10,6GHz è stato
regolamentato negli USA e poco dopo anche in Europa e Giappone, quest'ultima
viene impiegata in particolare per la trasmissione a corto raggio e la
localizzazione di precisione.
Tali applicazioni richiedono una bassa potenza di trasmissione e dunque,
circuiti che realizzano radio a impulsi UWB si prestano bene ad essere integrati
tramite tecnologie a basso costo come lo standard CMOS.\@

In questa tesi si analizzano e dimensionano due oscillatori LC per 
trasmettitori UWB in topologia \emph{cross-coupled} operanti alla frequenza di
8GHz e realizzati in tecnologia CMOS da $0,13 \mu m$.
Tali oscillatori differiscono esclusivamente nella soluzione adottata per l'
accensione e lo spegnimento del circuito tramite un segnale di controllo
esterno atto a generare gli impulsi.

Utilizzando un simulatore circuitale \emph{transistor level} ed alcuni modelli 
teorici, viene analizzato l' effetto sul comportamento del circuito esercitato
dalle variazioni apportate al dimensionamento dei componenti, ponendo
particolare attenzione all'efficienza energetica della trasmissione.

Si propone infine la soluzione che massimizza la potenza erogata all'antenna
rispetto a quella dissipata nel circuito dai componenti parassiti.
\end{abstract}
\end{document}
