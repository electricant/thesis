% Document Class and packages
\documentclass[a4paper]{article}
\usepackage{subfiles}
\usepackage[utf8]{inputenc}
\usepackage[italian]{babel}
\usepackage[usenames,dvipsnames,svgnames,table]{xcolor}
\usepackage{geometry}

% about:author
\author{Paolo Scaramuzza}
\title{Studio dell'efficienza di oscillatori LC integrati per impulsatori 
	Ultra Wideband}
\date{} % HACK: avoids date in the title

\begin{document}
\maketitle
%%%%%%%%%%%%%%%%%%%%%%%%%%%%%
% Abstract
%%%%%%%%%%%%%%%%%%%%%%%%%%%%%
\subfile{tex/abstract}
%%%%%%%%%%%%%%%%%%%%%%%%%%%%%
% Introduzione
%%%%%%%%%%%%%%%%%%%%%%%%%%%%%
\section{Introduzione}
\begin{itemize}
	\item Riprendere ed espandere il discorso del primo paragrafo
		dell'abstract citando la fonte delle leggi che hanno regolamentato
		la banda tra 3.1 e 10.6GHz.
		(W. Hirt, The european UWB radio regulatory \ldots)
	
	\item Citare possibili applicazioni, rifacendosi ad articoli riguardo
	\begin{itemize}
		\item Localizzazione di precisione (T.\ Terada, S.\ Yoshizumi et.\
			al.\, A CMOS impulse radio ultra-wideband transceiver for
			1Mb/s data communications and $\pm 2.5cm$ range findings)
		\item Trasmissione (S. Gambini, J. Crossley, E. Alon and J. Rabaey,
			A fully integrated 290pJ/bit UWB dual-mode \ldots)
			(Neviani, vari)
	\end{itemize}

	\item Dire perché si usa una topologia \emph{cross-coupled}, presentarne
		lo schema e spiegarne in breve il funzionamento
	\item Accennare ai contributi apportati e al fatto che simulazioni e
		teoria sono concordi
	\item Due parole sulla struttura dell'elaborato
\end{itemize}
%%%%%%%%%%%%%%%%%%%%%%%%%%%%%
% Topologie e parametri
%%%%%%%%%%%%%%%%%%%%%%%%%%%%%
\section{Topologie e parametri di interesse}
\begin{itemize}
	\item Descrizione delle topologie: Tipo 1 e Tipo 2 (con invertitori)
	\item Modello del trasformatore (cosa si è trascurato e perché)
	\item Presentazione del modello semplificato del circuito in rapporto 	
		all'oscillatore vero e proprio (Dimostrazione sintetica di come si
		giunge dal modello con trasformatore al solo generatore connesso ad
		un partitore di corrente)
	\item Discussione dei parametri di interesse e effetto delle
		loro variazioni dal punto di vista teorico (qualche grafico)
\end{itemize}
%%%%%%%%%%%%%%%%%%%%%%%%%%%%%
% Risultati numerici
%%%%%%%%%%%%%%%%%%%%%%%%%%%%%
\section{Risultati numerici}
\begin{itemize}
	\item Descrizione delle simulazioni in regime transitorio
	\item Discussione dei risultati ottenuti in rapporto a quelli teorici
	\item Presentazione dei parametri del trasformatore che garantiscono
		la massima efficienza e confronto con la simulazione del
		trasformatore da solo (differiscono dell' $ 1\% $)
	\item Interpretazione dei precedenti risultati
	\item Confronto dell'efficenza delle due topologie 
		($ 29.4\% $ tipo 1 rispetto $ 28.1\% $ tipo 2)
	\item Confronto tempi di accensione e spegnimento quando vengono impulsati
	\item Dire che il tipo 2, vista la differenza trascurabile tra le due
		figure di efficienza, può essere migliore dal punto di vista 
		dell'impiego di area perché il PMOS è di dimensioni minime mentre 
		ho due NMOS di dimensioni comparabili a quelli dell'oscillatore 
		($ 280\mu m $) e non 3 da $ 260\mu m $
		MOS più piccoli inoltre riducono l' energia disspiata dalla
		circuiteria di controllo per accendere e spegnere l' oscillatore
\end{itemize}
%%%%%%%%%%%%%%%%%%%%%%%%%%%%%
% Conclusione
%%%%%%%%%%%%%%%%%%%%%%%%%%%%%
\section{Conclusione}
Riassumere il lavoro e i risultati ottenuti in una mezza pagina.
\subsection{Sviluppi futuri}
\begin{itemize}
	\item Dal momento che non si è analizzato il layout e l'efficienza delle
		due topologie è molto simile può essere interessante valutare la 
		minima area occupata da entrambe
	\item Impatto dal punto di vista dell' efficienza della minore capacità
		dei MOS
	\item Un migliore modello di trasformatore può garantire un'accuratezza
		migliore nei risultati della simulazione
	\item Si è valutato il consumo di potenza solo ad oscillatore attivo.
		Sarebbe da valutare anche il consumo in funzionamento impulsato
\end{itemize}
\end{document}
